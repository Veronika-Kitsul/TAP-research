\section{Introduction}
\label{sec:intro}

The Internet of Things (IoT) devices are constantly increasing in their popularity with more than 10 billion
active devices in 2021~\cite{DBLP:webpage/Bojan/IoTstats}. As people around the world start using IoT
in the context of smart homes more and more, the ability to connect devices to other digital services is a
very helpful automation technique that can do small and monotonous tasks for the user to save time. To 
expand their network and support more services, IoT devices are oftentimes integrated with third-party
services called Trigger-action systems~\cite{DBLP:journals/access/XuZZCDG19, DBLP:conf/chi/UrHBLMPSL16}.

Trigger-action platforms (TAPs) comprise several parts: \emph{services}, \emph{platforms}, and
\emph{applets}. A client can install an applet that connects two services together on a platform. For
example, suppose the client wears a Fitbit to track their sleep duration. If the Fitbit records a sleep
duration under some target duration, the platform adds a reminder to the client's Google Calendar to 
sleep more the next day. The platform connects the Fitbit and Google Calendar services together 
through the applet ``If sleep is under a target duration, then add a reminder to
Google Calendar.''
%\kl{This 
%example is from the Walnut paper~\cite{DBLP:journals/corr/abs-2009-12447}, but we should come with
%another, more compelling example by analyzing more applets real people use.} 

One of the first and biggest such TAPs is IFTTT ("If This, Then That") that works with
over 700 services and more than 20 million users around the world ~\cite{ifttt-website, DBLP:conf/sp/ChenCWSCF21}. 
Because IFTTT has an access to the vast number of devices and users' data, hacking it presents a
bigger potential threat than if only one of your devices or digital services would be hacked. TAPs
compute over sensitive data and run millions of actions on their users' behalf. Many scholars have been
examining IFTTT-like services for years to point out their numerous security
and privacy flaws and suggest better practices (~\cite{DBLP:conf/sp/ChenCWSCF21, DBLP:journals/corr/abs-2009-12447, DBLP:conf/imc/MiQZW17, DBLP:journals/corr/FernandesRJP17, DBLP:conf/www/SurbatovichABDJ17, DBLP:journals/access/XuZZCDG19}). 
However, even if there are ways to be more secure, private companies would not employ them because 
they \emph{would} like to know everything about you as a user.


Today's TAPs learn much information about clients:
\begin{itemize}[leftmargin=*]
  \item the services (both trigger and action, for example Fitbit and Google Calendar),
  \item the applet semantics (``If sleep is under a target duration, then add a
    reminder to Google Calendar.''),
  \item inputs to the applet (sleep duration), and
  \item outputs to the action service (the reminder).
\end{itemize}


Privacy-preserving TAPs (e.g., Chen et al.~\cite{DBLP:conf/sp/ChenCWSCF21} and 
Walnut~\cite{DBLP:journals/corr/abs-2009-12447}) propose using multi-party computation (MPC) to hide
inputs to the applet from the TAP. However, prior work doesn't cover any of the abovementioned 
sources of leakage, which is a substantial fault. Ideally, the TAP should learn no information about the
client (besides timing information). In this project, we ask if such a TAP design is possible, and if
so, what are the costs. 
