\section{Introduction}
\label{sec:intro}

Trigger-action platforms (TAPs), such as IFTTT, comprise several parts:
\emph{services}, \emph{platforms}, and \emph{applets}. A client can install an
applet that connects two services together on a platform. For example, suppose
the client wears a Fitbit to track their sleep duration. If the Fitbit records a
sleep duration under some target duration, then the platform adds a reminder to
the client's Google Calendar to sleep more the next day. The platform connects
the Fitbit and Google Calendar services together through the applet ``If sleep
is under a target duration, then add a reminder to Google Calendar.'' \kl{This
  example is from the Walnut paper~\cite{DBLP:journals/corr/abs-2009-12447}, but
  we should come with another, more compelling example.}

Today's TAPs learn quite a bit of information about clients:
\begin{itemize}[leftmargin=*]
  \item the services (Fitbit and Google Calendar),
  \item the applet semantics (``If sleep is under a target duration, then add a
    reminder to Google Calendar.''),
  \item inputs to the applet (sleep duration), and
  \item outputs to the action service (the reminder).
\end{itemize}

Privacy-preserving TAPs (e.g., Chen et al.~\cite{DBLP:conf/sp/ChenCWSCF21} and
Walnut~\cite{DBLP:journals/corr/abs-2009-12447}) propose using multi-party
computation (MPC) to hide inputs to the applet from the TAP. However, prior work
doesn't cover these other sources of leakage, which are substantial. Ideally,
the TAP should learn no information about the client (besides timing
information). In this project, we ask if such a TAP design is possible, and if
so, what are the costs?
