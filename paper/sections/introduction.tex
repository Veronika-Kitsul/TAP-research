\section{Introduction}
\label{sec:intro}

Internet of Things (IoT) devices are becoming ubiquitous, with more than ten
billion active devices as of 2021~\cite{DBLP:webpage/Bojan/IoTstats}. As users
increasingly use IoT devices in their smart homes, the ability to connect and
configure differences devices together to perform small tasks is a useful
automation technique. To facilitate this automation, trigger-action systems
offer a platform for connecting and integrating third-party
services~\cite{DBLP:journals/access/XuZZCDG19, DBLP:conf/chi/UrHBLMPSL16}. One
of the first and most popular trigger-action services is IFTTT (``If This Then
That''), which works with over 700 services and more than 20 million users
worldwide~\cite{ifttt-website, DBLP:conf/sp/ChenCWSCF21}.

Trigger-action systems comprise several parts: \emph{services}, the
\emph{trigger-action platform} (TAP), and \emph{recipes}. Clients can specify
and install recipes that connect services together and automatically execute on
the TAP. For example, suppose a client wants to receive immediate SMS
notifications for urgent emails. The client can connect their email and SMS
providers together with the recipe ``If I receive an email with \emph{urgent} in
the subject line, then send the email information to my smartphone via SMS.''
The TAP will receive the client's emails, check the subject lines for the
\emph{urgent} keyword, and send an SMS to the client's smartphone if so.
%
%If the Fitbit records a sleep
%duration under some target duration, the platform adds a reminder to the
%client's Google Calendar to sleep more the next day. The platform connects the
%Fitbit and Google Calendar services together through the recipe ``If sleep is
%under a target duration, then add a reminder to Google Calendar.'' \kl{TODO: New
%  example.}
%\kl{This 
%example is from the Walnut paper~\cite{DBLP:journals/corr/abs-2009-12447}, but we should come with
%another, more compelling example by analyzing more applets real people use.} 

TAPs have access to a vast number of users' devices and sensitive information,
so a data breach could lead to severe consequences. In fact, many researchers
have studied trigger-action systems for years to point out their numerous
security and privacy flaws and suggest better
practices~\cite{DBLP:conf/sp/ChenCWSCF21, DBLP:journals/corr/abs-2009-12447,
  DBLP:conf/imc/MiQZW17, DBLP:journals/corr/FernandesRJP17,
  DBLP:conf/www/SurbatovichABDJ17, DBLP:journals/access/XuZZCDG19}. However,
even with security countermeasures in place, the privacy risks remain. This has
prevented many privacy-conscious companies from integrating with trigger-action
systems, because TAPs can harvest sensitive information about their users.

For example, today's TAPs learn at least the following information about clients:
\begin{itemize}
  \item the trigger and action services (the client's email and SMS providers),
  \item the recipe semantics (``If I receive an email with \emph{urgent} in the
    subject line, then send the email information to my smartphone via SMS.''),
  \item inputs to the recipe (email subject lines), and
  \item outputs to the action service (the client's phone number).
\end{itemize}

In light of these privacy risks, researchers have proposed privacy-preserving
TAPs (e.g., Chen et al.~\cite{DBLP:conf/sp/ChenCWSCF21} and
Walnut~\cite{DBLP:journals/corr/abs-2009-12447}), which use multi-party
computation (MPC) to hide recipe inputs from the TAP. However, prior work does
not cover all of the abovementioned sources of leakage and has considerable
performance overhead. Ideally, the TAP should learn no information about the
client (besides timing information) and should be able to scale to the volume of
users and recipes executed in modern TAPs. In this paper, we present \sys, a
privacy-preserving trigger-action platform that combines privacy-enhancing
technologies, such as end-to-end encryption, multi-party computation, anonymous
tokens, and anonymous proxies, in such a way that leads to low performance
overhead. We find that \sys supports most commonly used TAP functionality today
and does so at low cost.

To summarize, our main contributions are the following:
\begin{itemize}
  \item We propose \sys, a trigger-action platform that achieves strong privacy
    and anonymity guarantees.
  \item We implement a prototype of \sys in Rust.
  \item We find that our \sys protype supports the most commonly used TAP
    functionality and does so with low performance overhead.
\end{itemize}
