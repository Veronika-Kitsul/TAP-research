\section{Conclusion}
\label{sec:conclusion}

In this work, we set out to build a trigger-action platform that is both
practically efficient and provides strong privacy and anonymity guarantees,
culminating in \sys. Starting from previous work on privacy-preserving
trigger-action platforms using MPC, we make several important observations that
allow us to replace expensive MPC computations with cheap, native executions. We
also extend previous work by proposing a new anonymity protocol that
additionally hides sensitive metadata (e.g., the services involved and recipe
semantics). We find that \sys can indeed support the most common functionality
used in today's TAPs with cheap cryptographic tools and can reserve expensive
MPC for special-case recipes. In sum, \sys offers a new point in the design
space that provides both practicality and privacy.

%This paper builds upon previous work in testing out security designs that would make trigger-action
%platforms more secure. Security here is based on the principle that if TAP knows nothing, then it is able
%to leak nothing. In reality, however, no system is perfectly secure, but we design a security scheme that
%would allow to hide as much as possible. The proposed system operates on Garbled Circuits protocol
%in the MPC library available on C++ in open source and functions as other TAPs on the market. But the
%TAP itself does not know about the information it is getting, both the trigger and action do not know each
%other, ELSE?
%
%Talk about our quantative results
