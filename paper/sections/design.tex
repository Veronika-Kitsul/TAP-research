\section{Design}
\label{sec:design}

Things we want to hide from the TAP (in order from most important to least
important):
\begin{itemize}[leftmargin=*]
  \item inputs to the applet (sleep duration),
  \item outputs to the action service (the reminder),
  \item the applet semantics (``If sleep is under a target duration, then add a
    reminder to Google Calendar.''), and
  \item the services (Fitbit and Google Calendar).
\end{itemize}\bigskip

Unavoidable leakage that we might be able to leverage:
\begin{itemize}[leftmargin=*]
  \item The trigger service (Fitbit) already knows the inputs to the applet
    (sleep duration).
  \item The action service (Google Calendar) will see the outputs (the
    reminder).
\end{itemize}\bigskip

How to hide the inputs to the applet?
\begin{itemize}[leftmargin=*]
  \item \emph{Multi-party computation (MPC).} The trigger service and TAP can
    run a 2-PC to hide the inputs from the TAP.
    \begin{itemize}
      \item Pros: General-purpose (can compute any function over the inputs) and
        achieves the desired notion of privacy (learns nothing about inputs
        except what can be inferred from output)
      \item Cons: Complicated and expensive, may require the client to be online
        periodically, doesn't seem like the right tool for the job since the TAP
        has no inputs, can't hide applet semantics from the TAP
    \end{itemize}
  \item \emph{Trigger service computation.} The trigger service computes the
    conditional (i.e., ``If sleep is under a target duration'').
    \begin{itemize}
      \item Pros: Simple and efficient, achieves the desired notation of
        privacy, maybe still possible to hide applet semantics from the TAP
      \item Cons: Requires trigger service to do work (significant change from
        how things are done today)
    \end{itemize}
\end{itemize}\bigskip

How to hide the outputs of the applet?
\begin{itemize}[leftmargin=*]
  \item \emph{Encrypt to action service.} Need to know more about how actions
    work. Are they static (i.e., send message to action service)? Are they
    dynamic (i.e., inject part of input into message to action service)?
    \begin{itemize}
      \item Note: This is expensive to do in MPC, but easy otherwise (each
        action service has public key, and client installs action rule with TAP
        that encrypts the message to the action service).
    \end{itemize}
\end{itemize}\bigskip

How to hide the applet semantics?
\begin{itemize}[leftmargin=*]
  \item \emph{Trigger service talks to action service directly.} Trigger service
    learns actions service, have to give OAuth tokens to trigger service (some
    trigger services might be sketchy), trigger service has to do more work.
  \item \emph{Trigger service computation + encrypt to action service.}  Client
    generates fresh key pair and installs public key with trigger service and
    secret key with action service. Trigger service encrypts true/false to
    action service, and proxies through TAP. Action service decrypts result,
    uses oblivious transfer to retrieve action output.
\end{itemize}\bigskip

How to hide the services?
\begin{itemize}[leftmargin=*]
  \item This is probably too hard (probably have to add proxies).
\end{itemize}
