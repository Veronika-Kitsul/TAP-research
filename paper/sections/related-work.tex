\section{Related work}
\label{sec:related}

\paragraph{eTAP.}
The $\mathsf{eTAP}$ system~\cite{DBLP:conf/sp/ChenCWSCF21} also aims to provide
privacy for trigger-action platforms. We follow their approach of executing
recipes using MPC, but find ways to reduce the amount of MPC required by
half. Moreover, we also develop techniques to unlink services and the installed
recipes from client identities.

\paragraph{Walnut.}
Walnut~\cite{DBLP:journals/corr/abs-2009-12447} is another system that aims to
provide privacy for trigger-action platforms using MPC, but using two
non-colluding servers instead of one. They also do not consider how to unlink
services and installed recipes from client identities.

\paragraph{TAP OAuth security.}
Several other
works~\cite{DBLP:journals/corr/FernandesRJP17,DBLP:conf/ndss/FernandesRJP18}
study the security risks of clients handing and storing OAuth tokens on
TAPs. The studies find that OAuth tokens are often overprivileged and TAP
compromise would lead to attackers being able to misuse OAuth tokens to
manipulate client devices and data. DTAP~\cite{DBLP:conf/ndss/FernandesRJP18} is
a TAP design that prevents untrusted TAPs from misusing compromised OAuth tokens.

\paragraph{Other TAP studies.}
There have been various other studies measuring TAP ecosystems, the most popular
recipes used, and their security and privacy
risks~\cite{DBLP:conf/sp/ChenCWSCF21,DBLP:conf/chi/UrHBLMPSL16,DBLP:conf/www/SurbatovichABDJ17,
  DBLP:conf/imc/MiQZW17,DBLP:journals/corr/abs-2110-00068}.

%Filter-Fuzz~\cite{DBLP:journals/access/XuZZCDG19}
