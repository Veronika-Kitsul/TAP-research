\section{Background}
\label{sec:background}

\subsection{Trigger-Action Systems}

Trigger-action systems have three main components: trigger service (TS), action service (AS), and
trigger-action platform (TAP). Users interact with it through their computer connected to the cloud or
smartphone **(that can be offline? need to see how ifttt does it)**~\cite{DBLP:conf/sp/ChenCWSCF21}.

(ifttt website and ~\cite{DBLP:conf/spChenCWSCF21}. Clients configure the system by specifying triggers ("if my Fitbit sends a sleep data")
and actions ("create a Google Calendar reminder to sleep more"). 

\subsection{Market Analysis}
IFTTT, Zapier, Microsoft Power Automate, Integromat, Automate.io, Zoho Flow, Microsoft Flow, Tasker, Workflow, Huginn, Elastic.io, Workato, Actiondesk, Scriptable. 

\subsection{Cryptography Behind the Design}

\subsubsection{Multi-Party Computation}

Secure multi-party computation (MPC) enables a computation within a group of devices without revealing
any participant's private information. MPC allows verifiable computation, meaning the participants then
may confirm the result of the function they agreed on computing. Since its invention by Andrew Yao in
1982, the topic has attracted many researchers and evolved from theory to an important practical tool for
all kinds of applications. [the book]


\subsubsection{Garbled Circuits}

The first Garbled Circuits Protocol was introduced by Yao a few years after his first publication on MPC. It remains the basis for some of the most efficient MPC implementations. Yao's Garbled Circuits operate on the idea that there is a function $F(x,y)$, party $P_1$ holds $x \in X$, and party $P_2$ holds $y \in Y$.