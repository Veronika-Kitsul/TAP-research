\section{Background}
\label{sec:background}

The Internet of Things (IoT) devices are constantly increasing in their popularity with more than 10 billion
active devices in 2021~\cite{DBLP:webpage/Bojan/IoTstats}. As people around the world start using IoT
in the context of smart homes more and more, the ability to connect devices to other digital services is a
very helpful automation technique that can do small and monotonous tasks for the user. To expand their
network and support more services, IoT devices are oftentimes integrated with third-party services called
Trigger-action systems~\cite{DBLP:journals/access/XuZZCDG19, DBLP:conf/chi/UrHBLMPSL16}.

Trigger-action systems have three main components: trigger service (TS), action service (AS), and
trigger-action platform (TAP). Users interact with it through their computer connected to the cloud or
smartphone **(that can be offline? need to see how ifttt does it)**~\cite{DBLP:conf/sp/ChenCWSCF21}.
The IFTTT ("If This, Then That") platform is one of the first and the most popular TAP that works with
over 700 services and more than 20 million users around the world (ifttt website and ~\cite{DBLP:conf/spChenCWSCF21}. Clients configure the system by specifying triggers ("if my Fitbit sends a sleep data")
and actions ("create a Google Calendar reminder to sleep more"). 

