\section{Background}
\label{sec:background}

\subsection{Trigger-action systems}

Trigger-action systems have three main components: trigger services (TS), action
services (AS), and the trigger-action platform (TAP). Clients interact with TAPs
through trusted computing devices (typically their
smartphones)~\cite{DBLP:conf/sp/ChenCWSCF21}. Clients configure ``recipes'' with
the TAP by specifying triggers (i.e., ``if my Fitbit sends a sleep data'') and
actions (i.e., ``create a Google Calendar reminder to sleep more''). \kl{TODO:
  Update example.}

\kl{TODO: Update rest.} TAPs do not require their users to have any technical
skills to build complicated automated functions for their everyday use, which is
one of the reasons for their popularity. In addition, they can find hundreds of
ready-made applets for their usage by simply typing the services they want to
integrate. The templates are published by both the developers to ease people's
lives and by the users themselves, who would like to share their applets with
community. Those are not known to be regulated in any way.

Some recipes, however, involve sensitive data. Surbatovich et al.\kl{cite}
describes a potential case, in which an applet of a type ``open the window if
the temperature is above a threshold'' could enable attackers to affect the
temperature to open the windows when nobody is in the house.

%add a citation to websites of community applets on different platforms

\subsection{Market analysis}
Today's market of trigger-action platforms is fairly extensive and includes many
alternatives to IFTTT.  There is only one comparative study of such applications
that looks at IFTTT, Zapier, Integromat, Microsoft Power Automate, and Parabola
(there is another study that compares only IFTTT and
Zapier)~\cite{DBLP:conf/icict2/AbdouEF21}. However, there are way more many TAPs
than that. Others include Automate.io, Zoho Flow, Tasker, Workflow, Huginn,
Elastic.io, Workato, Actiondesk, Scriptable.

Platforms' capabilities also depend on the pricing plan the user is at. For
example, Zapier gives an opportunity to make 100,000 tasks per month with
unlimited multi-step applets and updates services each minute in a Company
plan. Conversely, free plan gives only 5 single-step applets with 100 tasks per
month and update time of 15 minutes.

Zapier, IFTTT, and Integromate all integrate messages, calls, GPS, Wi-Fi,
photos, and contacts. Abdou et al.~\cite{DBLP:conf/icict2/AbdouEF21} writes that
``they can send and save SMS messages, check the incoming class and give more
information about the caller, save the numbers you call, automate actions based
on phone location, record motion of the phone, track locations, get
notifications when employees connect or disconnect to the Wi-Fi, turn off Wi-Fi
automatically, backup photos, send/upload contacts from/to your phone.'' Zapier
also integrates with over 2000 applications as major technology companies are
interested in partnering with it.
% why????

All platforms, however, have one thing in common---security vulnerability. By
attacking only one digital automation platform, hackers may extract information
from all integrated applications and devices that may expose users to physical
or verbal threats. Sometimes, mistake on users behalf may lead to unexpected
behaviour that would leak sensitive information. Therefore, this paper is trying
to minimize the amount of data TAP has an access to so that when its security is
compromised, attackers can gain as little data as possible.

\subsection{Cryptographic building blocks}

\paragraph{Public-key encryption.}

Public key encryption is used to create secure communication channels between
parties. In a public-key encryption scheme, the recipient generates a pair of
keys $(\sk, \pk)$, where $\sk$ is the secret key and $\pk$ is the public key. A
sender can encrypt a message $m$ under the recipient's public key to obtain a
ciphertext $c = \enc(\pk, m)$. The recipient can decrypt the ciphertext with its
secret key $\sk$ to recover the message $m = \dec(\sk, c)$.

\paragraph{Multi-party computation.}

Multi-party computation (MPC) is a protocol that allows a group of mutually
distrusting parties to jointly compute some function while preserving certain
security properties, even if some parties are malicious (e.g., they may try to
learn private information or cause the result of the computation to be
incorrect). Two of the most basic security requirements of MPC, then, are
\emph{correctness} and \emph{privacy}.

Suppose there is a group of $n$ parties $P_1, \ldots, P_n$, where each party
$P_i$ holds a private input $x_i$. The parties wish to jointly compute some
agreed upon function $f(x_1, \ldots, x_n)$ over their private inputs (e.g.,
$x_i$ could be each party's salary and the function $f$ computes the average
salary). Correctness ensures that jointly computing $f(x_1, \ldots, x_n)$
returns the actual average, even in the face of adversarial behavior (parties
can, of course, still lie about their salaries). Privacy ensures that each party
involved in the protocol learns nothing about the private inputs of other
parties, besides what can be inferred from the output of the function (e.g., a
party involved in a two-party computation of the ``average'' function can still
infer the other party's input).

It is also important to specify what ``adversarial'' or ``malicious'' behavior
entails. In the semi-honest or ``honest-but-curious'' setting, all parties
followed the prescribed protocol faithfully, but passively try to learn more
information than intended. In the malicious setting, parties can arbitrarily
deviate from the prescribed protocol (correctness and privacy are, of course,
more difficult to achieve in this setting).

\paragraph{Yao's garbled circuits.}
Yao's garbled circuits is a two-party computation protocol that allows two
mistrusting parties to jointly compute a function over their private inputs. The
function is represented as a boolean circuit known to both parties, which we
will call the \emph{garbler} and the \emph{evaluator}. The garbler first
``garbles'' (i.e., encrypts) the circuit and sends the garbled circuit to the
evaluator along with the garbler's encrypted input. The garbler also generates
encrypted inputs for the evaluator, which the evaluator can privately retrieve
using a protocol known as oblivious transfer (the evaluator learns its garbled
input but the garbler learns nothing). Now that the evaluator has the garbled
circuit and both garbled inputs, it evaluates the garbled circuit to obtain the
output of the function, which is shared with the evaluator. This basic protocol
is correct and private in the semi-honest setting, which will turn out to be
sufficient for our purposes.

%Secure multi-party computation (MPC) enables a computation within a group of
%devices without revealing any participant's private information. MPC allows
%verifiable computation, meaning the participants then may confirm the result of
%the function they agreed on computing. Since its invention by Andrew Yao in
%1982, the topic has attracted many researchers and evolved from theory to an
%important practical tool for all kinds of applications
%~\cite{DBLP:journals/ftsec/EvansKR18}.
%
%None of the parties trusts each other nor do they trust any third party. We use
%the two-party scenario (also called 2PC) and set it up between user---TAP. MPC
%assumes direct secure channels between each pair and denotes encryption and
%decryption of a message $m$ under key $\kappa$ as $\mathsf{Enc}_\kappa(m)$ and
%$\mathsf{Dec}_\kappa(m)$ with the goal to learn the correct output of a mutual
%function without revealing private inputs.
%
%Our MPC will also use semi-honest adversary model. Semi-honest model corrupts
%parties but follows the protocol. However, the corrupt parties may try to learn
%as much as possible from the messages they receive. They cannot take any actions
%other than observe protocol execution, so they are also called
%\emph{honest-but-curious}.
%
%Under semi-honest adversaries, a protocol $\pi$ securely executes $\phi$ if
%there exists a simulator $\mathsf{Sim}$ such that for every subset of corrupt
%parties $C$ and all inputs $x_1, ..., x_n$ the distributions under real-ideal
%paradigm
%\begin{gather*}
%  \mathsf{Real}_{\pi}(\kappa, C, x_1, ..., x_n) \\
%  \mathsf{Ideal}_{\phi, \mathsf{Sim}} (\kappa, C, x_1, ..., x_n)
%\end{gather*} 
%are indistinguishable in $\kappa$.
%
%The first garbled circuits protocol was introduced by Yao a few years after his
%first publication on MPC. It remains the basis for some of the most efficient
%MPC implementations. Yao's Garbled Circuits operate on the idea that there is a
%function $F(x,y)$, party $P_1$ holds $x \in X$, and party $P_2$ holds $y \in Y$.
%Yao's garbled circuits protocol (GC) is the most widely known MPC technique that
%is also best performing. It is not associated with communication complexity but
%runs in constant rounds and avoids costly latency.

\paragraph{Anonymous tokens.}
An anonymous token scheme is a two-party protocol involving a client and a
server. The client interacts with the server to obtain tokens, which can later
be redeemed with the server (e.g., to anonymously authenticate itself for some
task). Importantly, the scheme should satisfy two core
properties---\emph{unlinkability} and \emph{unforgeability}. Unlinkability
ensures that the server cannot link a token's redemption with its respective
issuance. Unforgeability ensures that a malicious client cannot create valid
tokens on its own without the server.

\paragraph{Anonymous proxies.}
An anonymous proxy is a server that makes communications untraceable. For
example, a client can communicate with a server via an anonymous proxy so that
the server cannot learn the client's identity (i.e., IP address). A commonly
used anonymous proxy is Tor, which is an onion routing network that encrypts
routing information in such a way that prevents linking the origin and
destination.
