\section{Implementation}
\label{sec:implementation}

We implement the privacy protocol of \sys (described in
Section~\ref{sec:privacy}) in \kl{X,XXX} lines of Rust code. We choose Rust for
(1) efficiency, which reduces computation costs and hence latency, and (2)
memory safety, which prevents low-level security vulnerabilities.

Our \sys prototype currently only supports the identity transformation for
$f(x_t, c_a)$, which is essentially end-to-end encryption between trigger and
action services. Fortunately, it turns out that this alone is sufficient for
supporting most commonly used TAP functionality, since our system design pushes
predicate computations to the trigger services. For encryption, we use the
hybrid public-key encryption (HPKE) scheme.

Our prototype includes three components:

\paragraph{Trigger service.} The trigger service listens to polls from the
    TAP. Upon being polled, the trigger service will process a predicate over
    trigger data, encrypt the trigger data to the action service, and send
    predicate output and encrypted trigger data to the TAP.
    
\paragraph{Trigger-action platform.} The TAP polls the trigger service,
    receives the predicate output and encrypted trigger data, and forwards the
    encrypted trigger data to the action service if the predicate output is 1.
    
\paragraph{Action service.} The action service receives and decrypts the
    encrypted trigger data from the TAP, which it can then use to execute the
    desired action.

%\paragraph{Trigger service code.}
%Trigger service listens to the connections from TAP on the loop. When it
%receives the connection, it processes the information to send, encrypts it, and
%writes it to the stream.
%
%\paragraph{TAP code.}
%TAP serves essentially as a channel for sending encrypted data from the trigger service to action service. 
%TAP initiates the connection with the trigger, and upon receiving the requested data, sends it directly to
%the action service. With such communication scheme, TAP learns virtually nothing about the data it is
%getting, while trigger and action services do not communicate directly and do not know about each other.
%
%\paragraph{Action service code.}
%Action service has all features of a server, as it listens to the incoming connections from TAP on the loop,
%decrypts the received data, and then implements the action it is asked to do.

%\subsection{Used tools}
%
%We implement a prototype of \sys in Rust. Chen et al. \cite{DBLP:conf/sp/ChenCWSCF21} in their work use Python, however, we provide some reasons why we think Rust is better here: 
%\begin{itemize}
%	\item very fast and memory efficient (about on the same speed level as C);
%	\item powerful features for combining many functions and libraries;
%	\item extensive support system and vast amount of available crates;
%	\item ensures high safety, the compiler pings every possible bug (prevention of references to freed memory, concurrent memory access, etc.);
%	\item gives the ability to claim ownership of specific pieces of data, nothing else will be able to modify it;
%	\item efficient succinct code;
%	\item number one choice among many computer science researchers.
%\end{itemize}
%
%\subsection{System components}
%\paragraph{Trigger service code.}
%Trigger service listens to the connections from TAP on the loop. When it receives the connection, it
%processes the information to send, encrypts it, and writes it to the stream.
%
%\paragraph{TAP code.}
%TAP serves essentially as a channel for sending encrypted data from the trigger service to action service. 
%TAP initiates the connection with the trigger, and upon receiving the requested data, sends it directly to
%the action service. With such communication scheme, TAP learns virtually nothing about the data it is
%getting, while trigger and action services do not communicate directly and do not know about each other.
%
%
%\paragraph{Action service code.}
%Action service has all features of a server, as it listens to the incoming connections from TAP on the loop,
%decrypts the received data, and then implements the action it is asked to do.
