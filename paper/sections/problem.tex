\section{Overview and goals}
\label{sec:problem}

In a trigger-action system, there is a client, $T$ trigger services, $A$ action
services, and a trigger-action platform (TAP). Each trigger service $t \in [T]$
exposes a set of predicates $\mathbf{P}_t$ that activate the trigger service. A
predicate $p : \mathcal{X} \times \mathcal{C} \to \{0,1\}$ takes as input
trigger data $x \in \mathcal{X}$ and user constant $c \in \mathcal{C}$, and
outputs a bit. Each action service $a \in [A]$ exposes a set of transformations
$\mathbf{T}_a$ that activate the action service. Each transformation $f :
\mathcal{X} \times \mathcal{C} \to \mathcal{Y}$ also takes as input trigger data
$x \in \mathcal{X}$ and user constant $c \in \mathcal{C}$, and outputs an action
$y \in \mathcal{Y}$.

To set up the basic recipe with the TAP, the client specifies:
\begin{itemize}
  \item a trigger service $t \in [T]$, a predicate $p \in \mathbf{P}_t$, and a
    trigger constant $c_t \in \mathcal{C}$;
  \item an action service $a \in [A]$, a transformation $f \in \mathbf{T}_a$,
    and an action constant $c_a \in \mathcal{C}$.
\end{itemize}
When the trigger service $t$ receives trigger data $x_t \in \mathcal{X}$, the
recipe executes as follows:
\[
  \text{if}~p(x_t, c_t) = 1\text{, then send}~f(x_t, c_a)~\text{to}~a.
\]

In today's TAPs, such as IFTTT, both trigger data and user constants are
supplied to the TAP in the clear, which executes the recipe. This leaks all
information, including the results of the computation.

In \textsf{eTAP}~\cite{DBLP:conf/sp/ChenCWSCF21}, recipes are executed using MPC
(i.e., $p(x_t, c_t)$ and $f(x_t, c_a)$ are executed in MPC), which achieves the
following security properties, stated informally:
\begin{itemize}
  \item the TAP does not learn the trigger data $x_t$, the user constants $c_t,
    c_a$, or the results of the computation;
  \item the trigger service does not learn the user constants $c_t, c_a$; and
  \item the action service does not learn the trigger data $x_t$, the user
    constants $c_t, c_a$, or the output $f(x_t, c_a)$ if $p(x_t, c_t) = 0$.
\end{itemize}

While \textsf{eTAP} achieves strong privacy guarantees, we explore another point
in the design space that we argue achieves strong privacy, but with much lower
cost and complexity. Our two key observations are the following:
\begin{itemize}
  \item Each trigger service pre-defines a fixed set of predicates. These
    pre-defined predicates are far more widely used than arbitrary user-defined
    predicates.
  \item User-supplied trigger constants $c_t$ are often not particularly
    sensitive to trigger services. Consider the predicate ``email subject line
    contains the word \emph{confidential},'' where the keyword
    \emph{confidential} is the trigger constant. In this example, the email
    trigger service sees all email subject lines (the trigger data) in the
    clear, so hiding the trigger keyword does not buy much (but certainly
    \emph{some}) in the way of privacy.
\end{itemize}
Given these observations, in our TAP design, we push the computation of the
predicate $p(x_t, c_t)$ to the trigger service---$p$ is not arbitrary
client-supplied code and $x_t$ and $c_t$ need not be hidden from the trigger
service. \kl{The computation of $p$ is also probably cheaper than encoding
  inputs for MPC. Also computing $p$ in MPC can be quite expensive (e.g., due to
  substring operations).}

The computation of the transformation $f(x_t, c_a)$, however, mixes data from
the trigger service ($x_t$) and data intended for the action service ($c_a$), so
this is executed using MPC on the TAP. To summarize the privacy properties we
are after:
\begin{itemize}
  \item The TAP shouldn't learn $x_t, c_t, c_a$ or the results of $f(x_t, c_a)$.
  \item The action service shouldn't learn $x_t$, except what can be inferred
    from the output of $f(x_t, c_a)$.
\end{itemize}

\kl{Simpler recipes:}
Some recipes are even simpler. The client specifies:
\begin{itemize}
  \item a trigger service $t \in [T]$, a predicate $p \in \mathbf{P}_t$, and a
    trigger constant $c_t \in \mathcal{C}$;
  \item an action service $a \in [A]$, a transformation $f \in \mathbf{T}_a$,
    and an action constant $c_a \in \mathcal{C}$.
\end{itemize}
When the trigger service $t$ receives trigger data $x_t \in \mathcal{X}$, the
recipe executes as follows:
\[
  \text{if}~p(x_t, c_t) = 1\text{, then send}~f(c_a)~\text{to}~a.
\]
For simpler recipes, the predicate can execute entirely on the trigger service
and the transformation can execute entirely on the action service. In this
setting, we ask whether we can hide more information from the TAP, such as the
rule semantics and the services involved.

\kl{Is it a bad idea if trigger services communicate directly with action
  services?}
\begin{itemize}
  \item Trigger services have access to and store OAuth tokens.
  \item Each trigger service essentially has to run a TAP.
\end{itemize}

\kl{Otherwise, can a client install a rule and have it execute anonymously?}
\begin{itemize}
  \item Authenticated client can buy anonymous token to install a rule.
  \item Client can anonymously install a rule by presenting anonymous
    token. \kl{How to handle OAuth tokens? TODO: Figure out how they work.}
\end{itemize}
